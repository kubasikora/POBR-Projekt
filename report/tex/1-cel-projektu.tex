Celem projektu jest praktyczne zapoznanie się studentów z cyfrowymi metodami przetwarzania, analizy i rozpoznawania obrazów. 

Dla obrazów zawierających logo restauracji \bk należy dobrać, zaimplementować i przetestować odpowiednie procedury wstępnego przetworzenia, segmentacji, wyznaczania cech oraz identyfikacji obrazów cyfrowych. Powstały w wyniku projektu program powinien poprawnie rozpoznawać wybrane obiekty dla reprezentatywnego zestawu obrazów wejściowych. W trakcie projektu należy przetestować wybrane algorytmy i ocenić ich praktyczną przydatność. Wnioski powstałe w trakcie projektu muszą zostać przedstawione w formie pisemnego sprawozdania. Zaliczenie projektu dokonywane jest na podstawie pokazu działania zrealizowanego programu oraz sprawozdania. Sprawozdanie ma zawierać wyszczególnienie wybranych i zaimplementowanych algorytmów oraz wnioski powstałe w trakcie implementacji i testowania programu.

Jako dane wejściowe muszą być wykorzystane: zdjęcia w postaci papierowej - wykonane własnoręcznie lub wybrane np. z książek i czasopism, które należy zeskanować; lub zdjęcia w postaci cyfrowej - uzyskane za pomocą aparatu cyfrowego. Danych wejściowych nie mogą stanowić obrazy uzyskane bezpośrednio cyfrowo tzn. np. z programów typu MS Paint, Corel Draw itp. Ponadto w projekcie nie można wykorzystywać funkcji bibliotecznych do przetwarzania, analizy oraz rozpoznawania obrazów.
