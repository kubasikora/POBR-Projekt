Przedstawione na rysunku \ref{fig:bklogo} logo restauracji \bk składa się z~czterech części:
\begin{itemize}
    \item czerwonego napisu \bk,
    \item żółtej bułki od burgera podzielonej na pół,
    \item niebieskiej paska okalającego napis,
    \item białego okrągłego tła.
\end{itemize}

\begin{figure}[tb]
    \centering
    \includegraphics[width=0.4\columnwidth]{./figures/bklogo.pdf}
    \caption{Logo restauracji \bk~\cite{WikipediaEN:bklogo}}
    \label{fig:bklogo}
\end{figure}

Każda z~części ma swoją stałą, rozróżnialną barwę, co pozwala ją w~pełni rozróżnić od innych elementów. Poszczególne elementy praktycznie nie nachodzą na siebie, nie licząc kawałka czerwonego napisu zachodzącego na fragment niebieskiej obwódki. Logo jest wpisane w~okrąg, dzięki czemu dobrze skaluje się w~przestrzeni.

Typowo, logo \bk można znaleźć na elewacjach budynków tej restauracji, na znakach przydrożnych oraz wewnątrz samej restauracji. Co więcej, logo może być przedstawione z~wewnętrznym podświetleniem lub bez niego. Powoduje to bardzo zmienne warunki oświetleniowe znaku, co mimo prostego zestawu barw, czyni z~niego ciekawy obiekt do automatycznego wykrywania.
