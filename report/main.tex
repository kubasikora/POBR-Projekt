\documentclass{article}
\pdfpagewidth=8.5in
\pdfpageheight=11in

\usepackage{POBRreport}
% Use the postscript times font!
\usepackage{times}
\usepackage{soul}
\usepackage{url}
\usepackage{xcolor}
\usepackage{polski}
\usepackage[polish]{babel}
\usepackage[utf8]{inputenc}
\usepackage[T1]{fontenc}
\usepackage[utf8]{luainputenc}
\usepackage[hidelinks]{hyperref}
\usepackage[utf8]{inputenc}
\usepackage{caption}
\usepackage{indentfirst}
\usepackage{graphicx}
\usepackage{amsmath}
\usepackage{booktabs}
\urlstyle{same}

\title{Przetwarzanie Cyfrowe Obrazów \\ Wykrywanie logo restauracji Burger King}

\author{
Jakub Sikora
\affiliations
nr albumu: 283418\\
\emails
jakub.sikora2.stud@pw.edu.pl
}

\newcommand{\bk}{
    Burger King
}
\newcommand{\todo}[1]{\textcolor{blue}{\textbf{TO DO:} #1}}

\begin{document}

\maketitle

\section{Treść zadania}
\label{sec:cel-projektu}
Celem projektu jest praktyczne zapoznanie się studentów z cyfrowymi metodami przetwarzania, analizy i rozpoznawania obrazów. 

Dla obrazów zawierających logo restauracji \bk należy dobrać, zaimplementować i przetestować odpowiednie procedury wstępnego przetworzenia, segmentacji, wyznaczania cech oraz identyfikacji obrazów cyfrowych. Powstały w wyniku projektu program powinien poprawnie rozpoznawać wybrane obiekty dla reprezentatywnego zestawu obrazów wejściowych. W trakcie projektu należy przetestować wybrane algorytmy i ocenić ich praktyczną przydatność. Wnioski powstałe w trakcie projektu muszą zostać przedstawione w formie pisemnego sprawozdania. Zaliczenie projektu dokonywane jest na podstawie pokazu działania zrealizowanego programu oraz sprawozdania. Sprawozdanie ma zawierać wyszczególnienie wybranych i zaimplementowanych algorytmów oraz wnioski powstałe w trakcie implementacji i testowania programu.

Jako dane wejściowe muszą być wykorzystane: zdjęcia w postaci papierowej - wykonane własnoręcznie lub wybrane np. z książek i czasopism, które należy zeskanować; lub zdjęcia w postaci cyfrowej - uzyskane za pomocą aparatu cyfrowego. Danych wejściowych nie mogą stanowić obrazy uzyskane bezpośrednio cyfrowo tzn. np. z programów typu MS Paint, Corel Draw itp. Ponadto w projekcie nie można wykorzystywać funkcji bibliotecznych do przetwarzania, analizy oraz rozpoznawania obrazów.


\section{Logo restauracji \bk}
\label{sec:logo-bk}
Przedstawione na rysunku \ref{fig:bklogo} logo restauracji \bk składa się z~czterech części:
\begin{itemize}
    \item czerwonego napisu \bk,
    \item żółtej bułki od burgera podzielonej na pół,
    \item niebieskiej paska okalającego napis,
    \item białego okrągłego tła.
\end{itemize}

\begin{figure}[tb]
    \centering
    \includegraphics[width=0.4\columnwidth]{./figures/bklogo.pdf}
    \caption{Logo restauracji \bk~\cite{WikipediaEN:bklogo}}
    \label{fig:bklogo}
\end{figure}

Każda z~części ma swoją stałą, rozróżnialną barwę, co pozwala ją w~pełni rozróżnić od innych elementów. Poszczególne elementy praktycznie nie nachodzą na siebie, nie licząc kawałka czerwonego napisu zachodzącego na fragment niebieskiej obwódki. Logo jest wpisane w~okrąg, dzięki czemu dobrze skaluje się w~przestrzeni.

Typowo, logo \bk można znaleźć na elewacjach budynków tej restauracji, na znakach przydrożnych oraz wewnątrz samej restauracji. Co więcej, logo może być przedstawione z~wewnętrznym podświetleniem lub bez niego. Powoduje to bardzo zmienne warunki oświetleniowe znaku, co mimo prostego zestawu barw, czyni z~niego ciekawy obiekt do automatycznego wykrywania.


\section{Zarys algorytmu przetwarzania}
\label{sec:algorytm}
\todo{Opisać zgrubnie jak przetwarzany będzie obraz, w celu wykrycia logo \bk. Przygotować schemat blokowy rozwiązania.}

\section{Przetwarzanie wstępne}
\label{sec:preprocessing}
\todo{Opisać algorytmy przetwarzania wstępnego. Dać źródła i opisać implementację.}

\subsection{Skalowanie obrazu}
\todo{Opisać algorytm skalowania obrazu - interpolacji dwuliniowej. Znaleźć źródła. Opisać inne algorytmy skalowania obrazu. }

\subsection{Filtracja zakłóceń}
\todo{Opisać algorytm filtracji zakłóceń - filtracja medianowa. Znaleźć źródła.}

\section{Konwersja przestrzeni barw}
\label{sec:przestrzenie}
\todo{Opisać przestrzenie barw RGB, BGR (openCV) oraz HSV. Porównać HSV z innymi tego typu na przykład HSL, HSI, YUV. Koniecznie dodać tutaj jakieś źródła.}



\section{Segmentacja}
\label{sec:segmentacja}
\input{tex/6-segmentacja.tex}

\section{Wyznaczenie cech}
\label{sec:wyznaczanie-cech}
\input{tex/7-wyznaczanie-cech.tex}

\section{Identyfikacja}
\label{sec:identyfikacja-cech}
\input{tex/8-identyfikacja-cech.tex}

\section{Podsumowanie}
\label{sec:podsumowanie}
\todo{Króciutkie podsumowanie że projekt był fajny i takie tam}

\bibliographystyle{abbrv}
\bibliography{bibliography}

\end{document}