\documentclass{article}
\pdfpagewidth=8.5in
\pdfpageheight=11in

\usepackage{POBRreport}
% Use the postscript times font!
\usepackage{times}
\usepackage{soul}
\usepackage{url}
\usepackage{xcolor}
\usepackage{polski}
\usepackage[polish]{babel}
\usepackage[utf8]{inputenc}
\usepackage[T1]{fontenc}
\usepackage[utf8]{luainputenc}
\usepackage[hidelinks]{hyperref}
\usepackage[utf8]{inputenc}
\usepackage{caption}
\usepackage{indentfirst}
\usepackage{graphicx}
\usepackage{amsmath}
\usepackage{booktabs}
\usepackage{subfig}
\usepackage{float}
\usepackage{pgfplots}
\usepackage{listings}
\usepackage{color}
\usepackage{siunitx}
\usepackage{graphicx}
\usepackage{tikz}

\renewcommand{\labelitemii}{$\circ$}
\urlstyle{same}

\title{Przetwarzanie Cyfrowe Obrazów \\ Wykrywanie logo restauracji Burger King}

\author{
Jakub Sikora
\affiliations
nr albumu: 283418\\
\emails
jakub.sikora2.stud@pw.edu.pl
}

\newcommand{\bk}{
    Burger King
}
\newcommand{\todo}[1]{\textcolor{blue}{\textbf{TO DO:} #1}}

\begin{document}

\maketitle

\section{Treść zadania}
\label{sec:cel-projektu}
Celem projektu jest praktyczne zapoznanie się studentów z cyfrowymi metodami przetwarzania, analizy i rozpoznawania obrazów. 

Dla obrazów zawierających logo restauracji \bk należy dobrać, zaimplementować i przetestować odpowiednie procedury wstępnego przetworzenia, segmentacji, wyznaczania cech oraz identyfikacji obrazów cyfrowych. Powstały w wyniku projektu program powinien poprawnie rozpoznawać wybrane obiekty dla reprezentatywnego zestawu obrazów wejściowych. W trakcie projektu należy przetestować wybrane algorytmy i ocenić ich praktyczną przydatność. Wnioski powstałe w trakcie projektu muszą zostać przedstawione w formie pisemnego sprawozdania. Zaliczenie projektu dokonywane jest na podstawie pokazu działania zrealizowanego programu oraz sprawozdania. Sprawozdanie ma zawierać wyszczególnienie wybranych i zaimplementowanych algorytmów oraz wnioski powstałe w trakcie implementacji i testowania programu.

Jako dane wejściowe muszą być wykorzystane: zdjęcia w postaci papierowej - wykonane własnoręcznie lub wybrane np. z książek i czasopism, które należy zeskanować; lub zdjęcia w postaci cyfrowej - uzyskane za pomocą aparatu cyfrowego. Danych wejściowych nie mogą stanowić obrazy uzyskane bezpośrednio cyfrowo tzn. np. z programów typu MS Paint, Corel Draw itp. Ponadto w projekcie nie można wykorzystywać funkcji bibliotecznych do przetwarzania, analizy oraz rozpoznawania obrazów.


\section{Logo restauracji \bk}
\label{sec:logo-bk}
Przedstawione na rysunku \ref{fig:bklogo} logo restauracji \bk składa się z~czterech części:
\begin{itemize}
    \item czerwonego napisu \bk,
    \item żółtej bułki od burgera podzielonej na pół,
    \item niebieskiej paska okalającego napis,
    \item białego okrągłego tła.
\end{itemize}

\begin{figure}[tb]
    \centering
    \includegraphics[width=0.4\columnwidth]{./figures/bklogo.pdf}
    \caption{Logo restauracji \bk~\cite{WikipediaEN:bklogo}}
    \label{fig:bklogo}
\end{figure}

Każda z~części ma swoją stałą, rozróżnialną barwę, co pozwala ją w~pełni rozróżnić od innych elementów. Poszczególne elementy praktycznie nie nachodzą na siebie, nie licząc kawałka czerwonego napisu zachodzącego na fragment niebieskiej obwódki. Logo jest wpisane w~okrąg, dzięki czemu dobrze skaluje się w~przestrzeni.

Typowo, logo \bk można znaleźć na elewacjach budynków tej restauracji, na znakach przydrożnych oraz wewnątrz samej restauracji. Co więcej, logo może być przedstawione z~wewnętrznym podświetleniem lub bez niego. Powoduje to bardzo zmienne warunki oświetleniowe znaku, co mimo prostego zestawu barw, czyni z~niego ciekawy obiekt do automatycznego wykrywania.


\section{Akwizycja obrazu}
\label{sec:akwizycja-obrazu}
Panująca pandemia koronawirusa sprawiła że nie udało mi się wykonać własnoręcznie zdjęć omawianego obiektu. Z~tego powodu, zdjęcia loga musiałem znaleźć w~internecie, wykorzystując do tego wyszukiwarkę \emph{Google Images}. Wszystkie znalezione przeze mnie zdjęcia zostały przycięte do formatu w~standardzie 4:3. Zdjęcia przedstawiają logo \bk zarówno na zewnątrz (na fasadzie budynku) jak i~wewnątrz, z~wewnętrznym podświetleniem oraz bez, a~także w~dzień oraz w~nocy.



\section{Zarys algorytmu przetwarzania}
\label{sec:algorytm}
W~zaproponowanym algorytmie wykrywania logo \bk, można wydzielić pięć odrębnych faz przetwarzania. Poszczególne fazy zostały zaprezentowane za pomocą schematu blokowego na rysunku~\ref{fig:algorithm-overview}.

\begin{figure}[h]
    \centering
    \includegraphics[width=0.6\columnwidth]{figures/algorithmOverview.pdf}
    \caption{Ogólny schemat blokowy algorytmu wykrywania logo \bk}
    \label{fig:algorithm-overview}
\end{figure}

W~pierwszej kolejności obraz zostanie przepuszczony przez moduł przetwarzania wstępnego. Jego zadaniem jest poprawa jakości obrazu oraz ujednolicenie jego rozmiaru.
Następnie obraz zostanie skonwertowany do odpowiedniej przestrzeni barw, tak aby móc przeprowadzić proces segmentacji obrazu. Dla każdego wykrytego segmentu, wyznaczone zostanie dla niego zestaw cech, na podstawie których przeprowadzony zostanie etap identyfikacji.

\subsection{Szczegóły implementacyjne}
Cały algorytm wykrywania został zaimplementowany przy pomocy języka C++. Program korzysta ze specjalnie przygotowanych klas i~narzędzi, realizujących podstawowe algorytmy przetwarzania obrazu, zagregowanych w~pakiecie \texttt{POBR}. Akwizycja obrazu, jego wyświetlanie, zapisywanie oraz przechowywanie w~pamięci zostało zrealizowane za pomocą biblioteki OpenCV~\cite{opencv}. Działanie aplikacji sterowane jest za pomocą szeregu parametrów, które są wczytywane z~pliku konfiguracyjnego w~formacie YAML.

\section{Przetwarzanie wstępne}
\label{sec:preprocessing}
Pierwszym krokiem algorytmu wykrywania logo \bk, jest przetwarzanie wstępne. Celem przetwarzania wstępnego jest zmniejszenie rzeczywistego rozmiaru obrazu, względna poprawa jego jakości oraz usunięcie zakłóceń.

\subsection{Skalowanie obrazu}
\todo{Do obrazków: wstawić tylko wycinek znaku tak żeby pokazać rozpikselowanie.}
Celem skalowania obrazu jest stworzenie nowego obrazu o~zmienionym rozmiarze, wykorzystując do tego obraz oryginalny. W~przypadku projektowanego systemu, obraz analizowany jest poddawany skalowaniu aby zmniejszyć jego rzeczywisty rozmiar, celem uproszczenia dalszych obliczeń.

Do zmiany rozdzielczości obrazu cyfrowego zwykle wykorzystuje się metody interpolacji. Algorytmy tego typu można podzielić na algorytmy nieadaptacyjne oraz adaptacyjne. Pierwsza grupa dokonuje interpolacji w~ustalony z~góry sposób, niezależnie od zawartości przetwarzanego obrazu. Algorytmy adaptacyjne zmieniają sposób przetwarzania pikseli, biorąc pod uwagę cechy aktualnie przetwarzanego fragmentu. Adaptacja pozwala na zwiększenie jakości wizualnej, zwiększając przy tym koszt obliczeniowy~\cite{swierczynski2008podwyzszanie}. 

W~ramach projektu przeanalizowałem działanie trzech algorytmów nieadaptacyjnych: 
\begin{itemize}
    \item interpolacja najbliższym sąsiadem,
    \item interpolacja dwuliniowa,
    \item interpolacja dwukubiczna.
\end{itemize}

\subsubsection{Interpolacja najbliższym sąsiadem}
\todo{Opisać algorytm interpolacji najbliższym sąsiadem, wstawić wyniki interpolacji, opisać działanie.}

\todo{zasada działania algorytmu}

\todo{prezentacja działania algorytmu - opis, zdjecie, komentarz}

\todo{wady i zalety algorytmu}

\subsubsection{Interpolacja dwuliniowa}
\todo{dopisac cos o wadach i zaletach}
Algorytm interpolacji dwuliniowej jest algorytmem nieadaptacyjnym, nieco bardziej zaawansowanym niż pokrewny algorytm najbliższego sąsiada. Wartość każdego piksela obrazu wynikowego jest obliczana na podstawie czterech sąsiednich punktów obrazu wejściowego~\cite{algorytmy:bilinear}.

W~pierwszym kroku algorytmu oblicza się w~którym miejscu w~obrazie wejściowym znajduje się rozpatrywany punkt obrazu wyjściowego. Dokonuje się tego poprzez obliczenie współczynników skalowania, zgodnie z~wzorami~\ref{eqn:wsp-skalowania-x}~i~\ref{eqn:wsp-skalowania-y}.

\begin{equation}
    \label{eqn:wsp-skalowania-x}
    r_{x} = \frac{\mathrm{width}_{input}}{\mathrm{width}_{output}}
\end{equation}

\begin{equation}
    \label{eqn:wsp-skalowania-y}
    r_{y} = \frac{\mathrm{height}_{input}}{\mathrm{height}_{output}}
\end{equation}

Na podstawie współczynników $r_{x}, r_{y}$ dla piksela $(i, j)$ obrazu wyjściowego oblicza się pozycję~$(x, y)$ w~obrazie wejściowym, wykorzystując zależność \ref{eqn:skalowanie}.

\begin{equation}
    \label{eqn:skalowanie}
    \begin{array}{ll}
        x = i \cdot r_{x} & y = j \cdot r_{y}
    \end{array} 
\end{equation}

Na podstawie informacji o~docelowym punkcie, wybiera się cztery najbliższe punkty $F_{0,0}, F_{0,1}, F_{1,0}, F_{1,1}$, podobnie jak na~\ref{fig:bilinear-result} (żółtym kolorem zaznaczono punkt obliczony z~wzorów~\ref{eqn:skalowanie}). 

\begin{figure}[h]
    \centering
    \includegraphics[width=0.6\columnwidth]{figures/bi2.png}
    \caption{Dobór sąsiednich punktów w~algorytmie interpolacji dwuliniowej~\cite{algorytmy:bilinear}}
    \label{fig:bilinear-example}
\end{figure}

Następnie trzykrotnie przeprowadza się interpolację pomiędzy punktami, najpierw dwa razy w~kierunku poziomym pomiędzy $F_{0,0}$~i~$F_{1,0}$ oraz $F_{0,1}$~i~$F_{1,1}$ i~ostatni raz pomiędzy wynikami poprzednich interpolacji, zgodnie ze wzorem~\ref{eqn:interpolacja}.
Proces ten należy powtórzyć dla każdej składowej koloru z~osobna~\cite{algorytmy:bilinear}.

\begin{equation}
    \label{eqn:interpolacja}
    \begin{array}{l}
        F_{a,0} = (1-a) \cdot F_{0,0} + a \cdot F_{1,0} \\
        F_{a,1} = (1-a) \cdot F_{0,1} + a \cdot F_{1,1} \\
        F_{a,b} = (1-b) \cdot F_{a,0} + b \cdot F_{0,1} \\
    \end{array} 
\end{equation}

W~stworzonym rozwiązaniu, algorytm interpolacji dwuliniowej jest implementowany przez  obiekty klasy \texttt{POBR::BilinearInterpolationResizer}. Wyniki działania tego algorytmu zostały przedstawione na rysunku~\ref{fig:bilinear-result}. \todo{brakuje komentarza na temat wyniku testu.}

\begin{figure}[h]
    \centering
    \subfloat[800x800px]{{\includegraphics[scale=0.16]{./figures/ikonka.jpg} }}%
    \qquad
    \qquad
    \subfloat[400x400px]{{\includegraphics[scale=0.16]{./figures/ikonka_rev.png} }}%
    \caption{Efekt działania algorytmu interpolacji dwuliniowej na przykładzie zmniejszania logo \bk z~rozmiaru 800x800px do 400x400px \todo{zmienić rysunek na taki w którym jest powiększenie pewnego fragmentu a nie cały obrazek}}
    \label{fig:bilinear-result}
\end{figure}

\todo{w tym miejscu wady i zalety tego algorytmu}

\subsubsection{Interpolacja dwukubiczna}
\todo{Opisać algorytm interpolacji dwukubicznej, wstawić wyniki interpolacji, opisać działanie.}

\todo{zasada działania algorytmu}

\todo{prezentacja działania algorytmu - opis, zdjecie, komentarz}

\todo{wady i zalety algorytmu}

\subsection{Filtracja zakłóceń}
\todo{Opisać algorytm filtracji zakłóceń - filtracja medianowa. Znaleźć źródła.}

\section{Konwersja przestrzeni barw}
\label{sec:przestrzenie}
Po przetwarzaniu wstępnym obrazu, kolejnym krokiem algorytmu wykrywania logo \bk była konwersja przestrzeni barw z~RGB do przestrzeni~HSV.

\subsection{Przestrzeń RGB}
Najbardziej znanym modelem przestrzeni barw jest model opisywany współrzędnymi RGB. Nazwa współrzędnych pochodzi od pierwszych liter nazw barw podstawowych: (\textbf{R})~czerwonej, (\textbf{G})~zielonej oraz (\textbf{B})~niebieskiej. Model ten bazuje na właściwościach ludzkiego oka. Wrażenie widzenia dowolnej barwy można uzyskać poprzez zmieszanie trzech wiązek światła \cite{jankowski1990elementy}.

Model RGB jest domyślnie wykorzystywany w~informatyce do przechowywania plików graficznych. Wykorzystywana w~rozwiązaniu biblioteka OpenCV, również bazuje na tym modelu, odwracając kolejność kolorów. Zamiast jako pierwszą przechowywać informację o kolorze czerwonym, pierwsza przechowywana informacja mówi o~kolorze niebieskim, tak jak zostało to przedstawione na rysunku~\ref{fig:bgr}. Tak odwróconą przestrzeń nazywa się przestrzenią BGR~\cite{opencv}.

\begin{figure}[h]
    \centering
    \includegraphics[width=\columnwidth]{./figures/opencv-matrix-bgr.png}
    \caption{Sposób reprezentacji obrazu w~przestrzeni BGR, wykorzystywanej w~OpenCV~\cite{opencv}}
    \label{fig:bgr}
\end{figure}

\subsection{Przestrzeń HSV}
Model HSV czerpie swoją nazwę od pierwszych liter angielskich nazw składowych: \textbf{H} - hue (barwa), \textbf{S} - saturation (nasycenie), \textbf{V} - value (wartość). Model ten znacznie różni się od poprzednio omawianego modelu. W~RGB, wszystkie trzy zmienne niosą jednocześnie informację na temat chrominancji (kolorze) i~luminancji (jasności) punktu. W~modelu HSV, tylko jedna składowa \textbf{V} niesie informację o~jasności piksela a~pozostałe dwie zmienne \textbf{H}~i~\textbf{S}~przechowują informację o~chrominancji piksela. Taki sposób reprezentacji znacznie ułatwia wykrywanie na zdjęciu obiektów o~danej barwie lecz o~różnym oświetleniu. Typowo, przestrzeń HSV przedstawia się za pomocą stożka, zaprezentowanego na rysunku~\ref{fig:hsv}.

\begin{figure}[h]
    \centering
    \includegraphics[width=0.6\columnwidth]{./figures/HSV_cone.jpg}
    \caption{Stożek przestrzeni barw HSV~\cite{WikipediaPL:hsvCone}}
    \label{fig:hsv}
\end{figure}   

Podobnym modelem do HSV jest model HSL (\textit{Hue, Saturation, Lightness}). Oba modele jednakowo definiują zmienną barwową H, natomiast różnią się w~definicji pozostałych zmiennych. Największe różnice zachodzą w~definicji jasności/wartości koloru. W~modelu HSV, wartość $0\%$ reprezentuje kolor czarny, natomiast wartość $100\%$ kolor w pełni nasycony. W~modelu HSL, w~pełni nasycone kolory mają wartość jasności $50\%$, natomiast kolory czarny i~biały mają odpowiednio $0\%$ oraz $100\%$. Mimo iż model HSL bardziej intuicyjnie przedstawia jasność koloru, uznałem że bardziej odpowiedni do automatycznego przetwarzania będzie model HSV.

\subsection{Implementacja konwersji}
Konwersja piksela w~modelu RGB $(R, G, B)$ do modelu HSV $(H,S,V)$ jest operacją punktową i~nie wymaga informacji na temat wartości innych pikseli. W~programie, konwersję przeprowadza obiekt klasy \texttt{POBR::BGR2HSVConverter}.  

Zaimplementowany algorytm konwersji bazuje na podstawowym algorytmie z~OpenCV~\cite{opencv-conversions}. Pierwszym jego krokiem jest obliczenie jasności $V$, zgodnie ze wzorem \ref{eqn:value}.
\smallskip
\begin{equation}
    \label{eqn:value}
    V = \max{(R, G, B)}
\end{equation}

Kolejnym krokiem algorytmu, jest obliczenie nasycenia koloru $S$, korzystając ze wzoru~\ref{eqn:saturation}.

\begin{equation}
    \label{eqn:saturation}
    S = \left\{ 
        \begin{array}{ll}
            0, & V = 0 \\
            \min{(R, G, B)}, & V \ne 0
        \end{array} 
        \right.
\end{equation}

Ostatnim krokiem algorytmu jest obliczenie barwy $H$~zgodnie z~wzorem~\ref{eqn:hue}.

\begin{equation}
    \label{eqn:hue}
    H = \left\{ 
        \begin{array}{ll}
            \frac{(G - B) * 60\si{\degree}}{V - \min{(R, G, B)}} + 60\si{\degree}, & V = R \\
            \frac{(B - R) * 60\si{\degree}}{V - \min{(R, G, B)}} + 120\si{\degree}, & V = G \\
            \frac{(R - G) * 60\si{\degree}}{V - \min{(R, G, B)}} + 240\si{\degree}, & V = B
        \end{array} 
        \right.
\end{equation}
\smallskip

Wartości zmiennych $V$~i~$S$ mieszczą się w~przedziale $[0; 255]$ natomiast wartości zmiennej $H$ są ograniczone przez przedział $[0; 360]$. Obrazy przechowywane są tablicy o~ośmiobitowym rozmiarze podstawowym \texttt{uchar}. Zmienna $H$ nie mieści się w~komórce o~tym rozmiarze, dlatego też zdecydowałem się na przechowywanie w~tablicy wartości $\frac{H}{2}$. Podział ten jest solidnym kompromisem pomiędzy dokładnością reprezentacji a~rozmiarem obrazu w~pamięci komputera, ponieważ umożliwia na zdefiniowanie 256 różnych wartości kanału, co przy trzech kanałach daje ponad 16 milionów kolorów. Jest on również domyślnie wykorzystywany w~reprezentacji obrazów w~modelu~HSV w~bibliotece OpenCV~\cite{opencv}.

\section{Segmentacja}
\label{sec:segmentacja}
\todo{Opisać na schemacie blokowym lub sekwencji jak działa rozwiązanie progowania kolorów. Opisać że nie wyszło i dlaczego nie wyszło.}

\todo{Opisać algorytm segmentacji obrazu - rozrost obszarów}

\todo{Opisać algorytm docelowy. Docelowy algorytm segmentacji będzie wykrywał okrągłe obiekty za pomocą transformaty Hougha a następnie wyznaczał segmenty na podstawie kolorów (może wykorzystać już istniejące progowanie).}

\section{Wyznaczenie cech}
\label{sec:wyznaczanie-cech}
\todo{Opisać algorytm wyznaczania cech segmentów oraz jakie cechy zostały wyznaczone. Podeprzeć się tym co wyznacza matlab w swojej aplikacji Image Region Analyzer}


\section{Identyfikacja}
\label{sec:identyfikacja-cech}
\todo{Opisać algorytm identyfikacji segmentów z wyznaczonymi cechami.}


\section{Podsumowanie}
\label{sec:podsumowanie}
\todo{podsumowanie że projekt był fajny i takie tam, wstawić rezultaty, wstawić link do githuba}



\bibliographystyle{abbrv}
\bibliography{bibliography}

\end{document}